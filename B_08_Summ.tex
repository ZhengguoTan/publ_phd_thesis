\chapter{Summary and Outlook} \label{Chp:sum}

\section{Summary}
This thesis presents novel methodological developments in two major areas that both extend real-time MRI techniques and applications which are based on highly undersampled radial FLASH acquisitions and iterative image reconstruction as a nonlinear inverse (\acs{NLINV}) problem. The first new method is the exploration of advanced radial sampling schemes, namely the use of asymmetric echoes and multi-echo acquisitions, while the second contribution relates to advanced reconstruction principles, namely model-based reconstruction techniques that directly estimate parameter maps from raw data.

An asymmetric echo samples only a portion of the full echo and thus shortens the achievable echo time and repetition time. When applied to real-time phase-contrast flow MRI, its use not only improves temporal resolution, but also allows for the addition of flow-compensation gradients to suppress intra-voxel phase dispersion. Short echo times and first-order flow compensation for in-plane movements have been proven crucial for studies of patients with more complex flow patterns such as due to aortic valve insufficiency and partial stenosis.

Multi-echo radial sampling schemes offer multiple new opportunities ranging from a further speed-up of real-time data acquisitions to the reconstruction of quantitative parameter maps for off-resonance signal contributions and \acs{T2s} relaxation times. Two types of multi-echo radial FLASH sequences, i.e.~multi-echo single-spoke and multi-echo multi-spoke techniques, were developed and evaluated with the use of phantom and human studies. When compared to multi-echo single-spoke radial FLASH, the multi-spoke variant supplies a faster k-space coverage and various view-ordering schemes and therefore turns out to be promising for improved spatial and/or temporal resolution.

Partly in response to the new radial acquisition modes, the iterative image reconstruction demands new solutions. Thus, the second major part of this thesis was the development of image reconstruction techniques that advances well beyond the conventional \acs{NLINV} method for real-time MRI. For example, in case of real-time phase-contrast flow MRI, the computation of the desired phase-contrast (velocity) map requires at least two measurements with different velocity encodings. Conventionally, these measurements are treated as independent streams and reconstructed separately by NLINV (or Fourier transformation when using classical \acs{ECG}-synchronized acquisitions). The reconstructed images are then subject to a phase-difference calculation to obtain a phase-contrast map. For the past \num{30} years this two-step approach has been the only technical solution. However, the phase-difference calculation is prone to arbitrary phase noise in no or low MR signal areas and may even hamper the lumen definition and flow quantification. Therefore, a novel model-based reconstruction technique was developed which jointly estimates the anatomical image, the phase-contrast (velocity) map, and a set of coil sensitivity maps. The direct estimation of the phase-contrast map from the sets of raw data in combination with Tikhonov regularization ensures zero phase in no or low MR signal areas, which is clearly superior in sharpening vessel lumen and reducing streaking artifacts, especially for highly undersampled radial FLASH acquisitions. The technique was compared to the two-step NLINV approach and validated using a numerical phantom (ground truth), an experimental flow phantom, and measurements of human aortic blood flow.


\section{Future Work}

\subsection*{Model-based Reconstruction}
The current version of the advanced model-based reconstruction technique for real-time phase-contrast flow MRI to a certain degree still depends on the chosen scaling mechanism which is derived from the definition of the complex-difference image. This may be problematic in two more extreme situations, namely for very small phase differences (velocities) and in the presence of velocity aliasing. The former situation may increase the scaling value to such a degree that the concomitant regularization strength is weakened and noise accumulates in the estimate. On the other hand, the latter situation, which may be caused by the simultaneous presence of very high and very low flow, may lead to arbitrarily high phases in velocity-aliased regions and thus decrease the scaling, which in turn underestimates the resulting velocities. These scaling issues need to be resolved in order to make the model-based reconstruction technique a robust clinical tool.

As another potential extension, the integration of advanced regularizations, i.e., $L^1$-wavelet \cite{2008_NLINV_L1} and total (generalized) variation \cite{2012_NLINV_TGV}, into the model-based reconstruction may turn out to be advantageous and should be investigated. These non-quadratic regularizations offer better noise suppression, but require more complicated implementations. On the other hand, a prerequisite of the $L^1$-norm regularization is that the image to be regularized must be sparse. Fortunately, the complex-difference map is sparse by itself, requiring no sparsifying transform, so that this favorable feature may be exploited in a modified model-based reconstruction technique.

Further, as discussed in \cref{Chp:mir-pc}, the model-based reconstruction may be implemented with with different spatial encodings for the flow-compensated and flow-encoded dataset. In principle, this idea may help to increase the spatial resolution as more lines in k-space are available for image reconstruction. Moreover, respective reconstruction may effectively double the temporal resolution when combining such acquisitions with a two-sided flow encoding scheme and a sliding window which shifts the reconstruction by only one dataset rather than two in all current methods.

Finally, the model-based reconstruction can be greatly accelerated via a multi-\acs{GPU} implementation to allow for extensive clinic trials. 

\subsection*{Multi-Echo Radial FLASH}
The developments described in \cref{Chp:multi-echo} primarily focused on multi-echo multi-spoke acquisitions with linear view ordering for the convenience of image reconstruction. However, other view-ordering schemes need to be further investigated. Imaging parameters such as flip angle, bandwidth, number of spokes and echoes, echo spacing, and potential contrasts need to be thoroughly studied as well. Moreover, because this sequence is compatible with asymmetric echoes, it needs to be evaluated whether such a combination may further speed up acquisitions, or offers specific applications with variable echo spacings. 

With the implementation of more and more complex gradient switching schemes, it may also become necessary to develop special gradient delay corrections and k-space off-resonance corrections to maintain or improve image qualities. For multi-echo image reconstructions, the NLINV variants also require further optimization with regard to the number of Newton steps, temporal regularization, and initialization. As an ultimate goal, quantitative parametric mapping of \acs{T2s} relaxation times and off-resonance frequencies should be accomplished by a suitable model-based reconstruction technique. 

Taken together, a successful validation of undersampled multi-echo multi-spoke acquisitions with respective image reconstructions could certainly advance the general concept of real-time MRI, not only with respect to temporal resolution but also with respect to novel (dynamic) contrasts exploiting the \acs{T2s}-weighted signal decay and off-resonance phase modulation. These factors have various clinical applications such as the identification of hemorrhage and calcification, access to tissue oxygenation, susceptibility-weighted imaging, and water-fat separation.
