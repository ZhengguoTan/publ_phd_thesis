\chapter{Introduction} \label{Chp:intro}

Magnetic resonance imaging (\acs{MRI}) is a non-invasive imaging modality that is extensively used in routine clinic. MR signals originate from the oscillations of hydrogen nuclei excited by a radio-frequency (\acs{RF}) wave in the presence of a strong and static magnetic field. As human bodies contain a large number of hydrogen atoms, MRI provides excellent image contrasts among soft tissues. When compared to X-ray computed tomography (CT) and positron emission tomography (PET), MRI needs neither ionizing radiations nor radioactive tracers. Moreover, MRI offers various indispensable imaging methods, e.g.~functional imaging that utilizes blood-oxygen-level dependent (\acs{BOLD}) contrast, phase-contrast flow quantification, and susceptibility-weighted imaging.

Technically, MRI acquires data in Fourier domain in a point-by-point manner, which is time consuming and thus problematic when imaging uncooperative patients (e.g.~pediatric) and physiological movements (e.g.~beating heart and speaking processes). Hence, tremendous endeavors have been made over the past decades to accelerate MRI. One of the successful techniques is \textit{parallel imaging}, which uses spatially-varying receiver coils placed around the imaged object to simultaneously receive MR signals. The received multi-coil data contains complementary information about the imaged object and thus allows for sub-Nyquist sampling, typically known as \textit{undersampling}. Other successful techniques are fast imaging pulse sequences, i.e., echo-planar imaging and non-Cartesian sampling schemes. The former enables large reductions of the measuring time while retaining the amount of data sampled, as it acquires multiple echoes with different spatial encodings per RF excitation. The latter, especially the \textit{radial} sampling scheme, has been of great interest due to its low sensitivity to motion and resistance to undersampling.

Specifically, advances in parallel imaging as a \textit{nonlinear} inverse (\acs{NLINV}) problem enables joint estimations of both the object image and the coil sensitivity maps, which is superior in supplying more accurate coil sensitivities and better images when compared to parallel imaging techniques that pre-calibrate coil sensitivity maps. The application of NLINV to highly undersampled radial fast low angle shot (\acs{FLASH}) acquisitions has achieved \textit{real-time} MRI at millisecond resolution to resolve physiological processes without the need of physiological triggering and data sorting. The radial sampling scheme initially proposed for real-time MRI, however, is limited to one full echo acquisition per RF excitation. Therefore, two essential variants of the radial sampling scheme, namely asymmetric echo and multi-echo techniques are developed in this thesis. 

\textit{Asymmetric echo}, also known as partial Fourier imaging, refers to the sampling of an incomplete echo. Thus, asymmetric echoes shorten the echo time, which is of great importance in studies where long echo times would lead to image artifacts (e.g.~signal loss). In particular, real-time phase-contrast flow MRI requires at least two measurements with different velocity encodings, so shorter echo times can potentially speed up acquisitions. Moreover, asymmetric echoes allow for the addition of flow-compensation gradients to further reduce intra-voxel phase dispersion.

The conventional \textit{multi-echo} radial FLASH utilizes bipolar readout gradients with the same amplitude and duration to sample echoes in a fly-back-and-forth manner. This strategy is inefficient with respect to k-space coverage and temporal resolution. Therefore, in this thesis, a multi-echo multi-spoke (\acs{MEMS}) radial FLASH sequence, i.e., capable of acquiring multiple echoes with different spatial encodings per RF excitation, is developed for the purposes of faster k-space sampling and better temporal and/or spatial resolution. This sequence, most importantly, can advance various applications of real-time MRI, e.g., dynamic quantitative \acs{T2s} mapping, water-fat separation, functional imaging, and quantitative susceptibility mapping.

Further, a \textit{model-based} image reconstruction, especially for real-time phase-contrast flow MRI, is developed in this thesis as well. Model-based image reconstruction emerges as a novel reconstruction technique that estimates maps of interest in MR signal models directly from the acquired data. In phase-contrast flow MRI, two reconstruction steps are typically involved. First, the measurements with different velocity encodings are treated as independent streams and reconstructed separately via parallel imaging. Second, the phase-difference calculation between the reconstructed images is used to obtain phase-contrast (e.g.~velocity) maps. The second step, however, induces severe random phase noise in no or low MR signal areas (e.g.~air and lung), which can hamper the lumen definition, especially in the case of highly undersampled acquisitions. To ameliorate the situation, a novel model-based reconstruction technique, which directly reconstructs the phase-contrast map from the measured datasets based on a proper signal modeling and hence ensures zero phase in the areas without MR signals, is proposed.


